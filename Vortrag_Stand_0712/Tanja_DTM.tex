\begin{frame}
\frametitle{Erstellen von Prädiktoren}
\framesubtitle{Filterung der Prädiktoren}
\begin{itemize}\itemsep12pt
\item Nach Erstellung der Grundwörter konnte gefiltert werden, welche Wörter häufig auftraten
\item Denkbare Filtermethoden für das Wörterbuch:
\begin{itemize}
\item Nur Wörter, die mind. $n$ Mal aufgetaucht sind 
\item Nur Wörter, die in mind. $p\%$ der Reviews verwendet wurden
\end{itemize} 
\item Anschließend Erstellung einer binären Document-Term-Matrix, die kodiert, welche Grundwörter in welchen Reviews auftauchten
\item Alternative: PCA um aussagekräftige \glqq Wörterachsen\grqq \,zu bestimmen. Kein sichtbarer Erfolg.
\end{itemize}
\end{frame}

\begin{frame}
\frametitle{Erstellen von Prädiktoren}
\framesubtitle{Document-Term-Matrix}
\begin{itemize}\itemsep12pt
\item Wörterbucherstellung aus den verbliebenen Grundwörtern\\
\begin{center}
\begin{tabular}{|c|c|c|c|c|}
\hline
		& Lemma 	& n	\\
\hline
144 	& .		& 654\\
143	 	& der	& 531\\
142 	& und	& 440\\
%141	 	& ,		& 413\\
%140	 	& ich	& 318\\
...		& ...	& ...\\
2		& dann & 10\\
1		& abschluss & 10\\
\hline
\end{tabular}
 \end{center}

%\item Erstellung der DTM unter Beachtung der numerischen Reihenfolge der $doc\_id$s (d.h. nicht alphabetisch)
\item DTM-Zeilen: Reviews 1 bis 439, DTM-Spalten: Grundwörter aus Wörterbuch
\item zusätzliche Spalten: doc\_id, review, type (Zielvariable) 
\item Kodierung: Wort kommt in der Review vor (1) oder nicht (0)
\end{itemize}
\end{frame}

\begin{frame}
\frametitle{Erstellen von Prädiktoren}
\framesubtitle{Document-Term-Matrix}
\begin{itemize}\itemsep12pt
\item Anfügen weiterer Prädiktoren mit Anzahlen pro Review der
\begin{itemize}
\item Wörter %$count\_of\_words$
\item Fragen %$count\_of\_questions$
\item Imperative %$count\_of\_imperatives$
\item Sätze %$count\_of\_sentences$
\item Nebensätze/Trennungen %$count\_of\_subordinated\_clauses\_and\_separations$
\item Satzzeichen %$count\_of\_punctuation$
(ggf. einzeln betrachtet)
\item Wortarten %$count\_of\_parts\_of\_speech$
\end{itemize} 
\item Weitere Variablenselektion möglich

\end{itemize}
\end{frame}

\begin{frame}
\frametitle{Erstellen von Prädiktoren}
\framesubtitle{Document-Term-Matrix}
\vspace{12pt}
\begin{tabular}{|c|c|c|c|c|c|c|c|c|c|}
\hline
...	&dass&  der 	& dies & ... & doc\_id	& review & words & ...& type\\
\hline
... &0& 1		& 0		&	...	& 1		& Die Un...&12 			& ...& Ge\\
...	& 0&0		& 0		&	...	& 2		& schnell...&7 			& ...& In\\
...	& 0&0		& 0		&	...	& 3		& Sehr g...&9 			& ...& Do\\
...	& 0&0		& 0		&	...	& 4		& Super ...&8 			& ...& In\\
...	& 0&1		& 0		&	...	& 5		& Hervor...&11 			& ...& In\\
...	& 0&1		& 0		&	...& 6		& Die Zu...&27 			& ...& In\\
...	& 0&1		& 0		&	...& 7		& untern...&26 			& ...& In\\
...	& 0&1		& 0		&	...& 8		& Schnell...&23 			& ...& Ge\\
...	& 0&0		& 0		&	...& 9		& Unkom...&3 			& ...& Do\\
...	& 0&1		& 0		&	...& 10	& ich ha...&39 			& ...& St\\
...	& 0&1		& 0		&	...	& 11	& Besten...&19 			& ...& St\\
...	& 0&0		& 0		&...	& 12	& Sehr f...&8 			& ...& In\\
...	& ...&...	&...	&	...	& ...  	& ...		&... 			& ...& ...\\
\hline
\end{tabular}


%\begin{center}
%	\includegraphics[scale=0.5]{DFGHJKLKJHGFFKLKJHGFDFZULKJHGFDFGHJKLKJHGF.png}
%\end{center}
\end{frame}
