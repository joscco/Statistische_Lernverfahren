%\documentclass{beamer}
%\usepackage[utf8]{inputenc}
 \usepackage[latin1]{inputenc} %  Alternativ unter Windows
% \usepackage[T1]{fontenc}
%\usepackage[ngerman]{babel}
%\usepackage[toc,page]{appendix}
%\usepackage{latexsym}
%\usepackage{amsmath,amssymb,amsthm}
%\usepackage{hyperref}

%\newcommand*{\quelle}{%
%	\footnotesize Quelle:
%}

%\useoutertheme{infolines}
%\beamertemplatenavigationsymbolsempty
%\setbeamertemplate{headline}{}

%\title{Seminar Statistische Lernverfahren}
%\subtitle{Klassifikation von Rezensionstypen}
%\author[T.G., M.H., A.K., M.L., T.N., J.S.]{Till Gr�fenberg, Matthias H�u�ler, Alexander Kohlscheen, Michael Lau, Tanja Niklas, Jonathan Schmitz}
%\date{12. Dezember 2019}
%\begin{document}
%\begin{frame}
%\thispagestyle{empty}

%\titlepage
%\end{frame}

\section{Problemstellung}
\begin{frame}
\frametitle{Problemstellung}
\begin{itemize}\setlength\parskip{12pt}
\item Ziel: Klassifizierung von Reviews in folgende Typen:
\begin{center}
\begin{tabular}{c|c|c}
Texttyp & introvertiert & extrovertiert \\
\hline 
emotional & stetig & initiativ\\
rational & gewissenhaft & dominant
\end{tabular}
\end{center}
\item Gegeben: 439 bereits klassifizierte Reviews
\end{itemize}
\end{frame}

\section{Erstellen von Pr�diktoren}
\begin{frame}
\frametitle{Erstellen von Pr�diktoren}
\begin{itemize}\itemsep12pt
\item Klassifikation sollte durch verwendete W�rter geschehen
\item Zur�ckf�hrung auf Grundw�rter notwendig
\item Benutzung verschiedener Packages in \texttt{R} bzw. \texttt{Python} erm�glichte verschiedene Verfahren
\end{itemize}
\end{frame}

\section{Lemmatisierung und Stemming}
\begin{frame}
\frametitle{Erstellen von Pr�diktoren}
\framesubtitle{Was ist Stemming?}
\begin{itemize}\setlength\parskip{12pt}
	\item Verfahren, mithilfe dessen man verschiedene Varianten eines Wortes auf ihren gemeinsamen Wortstamm zur�ckf�hrt
	\item Durch Abschneiden von Pr�-/In- und Suffixen und Ersetzen von Umlauten, Diphtongen etc. Erzeugen von Wortst�mmen
	\item Beispiele: \\
	gelernt $\rightarrow$ lernen; \\
	Wohnungen $\rightarrow$ Wohnung
\end{itemize}
\end{frame}

\begin{frame}
\frametitle{Erstellen von Pr�diktoren}
\framesubtitle{Stemming}
\begin{itemize}\itemsep12pt
\item Eigene Implementierung nach Vorgabe von COMPEON in \texttt{R}
\item F�r Englische Sprache bereits vorgefertigte Tools z.B. 
\begin{itemize}
\item \texttt{porterstemmer} von \texttt{nltk} in Python
\item \texttt{snowballstemmer} von \texttt{nltk} in Python
\end{itemize}
\end{itemize}
\end{frame}


\begin{frame}
\frametitle{Erstellen von Pr�diktoren}
\framesubtitle{Was ist Lemmatisierung?}
\begin{itemize}
\item Das Lemma ist im Bereich der Linguistik die Grundform eines Wortes $\rightarrow$ Wortform z.B. in einem Nachschlagewerk
\item Zur�ckf�hrung auf grammatikalische Grundformen
\item Lemmatisierung als ein lexikonbasiertes Stemmingverfahren
\item Auftretende Probleme des Vorgangs:
\begin{itemize}
\item Ambiguit�ten
\item Wahl des Lemmas eines Wortes (Verbinfinitiv vs. Nomen)
\item Kompositazerlegung nicht eindeutig (Beispiel: Wachstube)
\item Simplizia (Beispiel: Kreuzer, Tangente)
\item Unregelm��igkeit von Verben im Deutschen
\end{itemize}
\end{itemize}
\end{frame}

\begin{frame}
\frametitle{Erstellen von Pr�diktoren}
\framesubtitle{Lemmatisierung}
\begin{itemize}\itemsep12pt
\item Erfordert vorgefertigte Packages z.B.
\begin{itemize}
\item \texttt{SpaCy} in Python
\item \texttt{nltk} in Python
\end{itemize}
\item Diese lieferten zus�tzlich Informationen �ber die Wortart
\item Auch hier f�r Englische Sprache ausgereifter als die deutsche Alternative
\end{itemize}
\end{frame}

\begin{frame}
\frametitle{Erstellen von Pr�diktoren}
\framesubtitle{Wortliste}
\begin{itemize}
\item Ausgangssituation: 439 Reviews �ber die Firma COMPEON 
\item 8792 W�rter in reviews\_preprocessed (mitunter mehrfach)
\end{itemize}
\begin{table}
	\centering
		\begin{tabular}[h]{l|l}
			cleaned\_text & preprocessed\_text \\
			\hline
			richtigen & richtig \\
			\hline
			darlehen & darleh \\
			\hline
			gew�nschte & wunsch \\
			\hline
			taggenuae & taggenua \\
			\hline
			gegen�ber & genub
\end{tabular}
		\item
	\caption{Wortliste Beispiele}
	\label{tab:WortlisteBeispiele}
\end{table}
\end{frame}

%\end{document}
