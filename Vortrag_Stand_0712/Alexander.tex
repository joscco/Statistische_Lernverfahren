\section{Problemstellung}
\begin{frame}
\frametitle{Problemstellung}
\begin{itemize}\setlength\parskip{12pt}
\item Ziel: Klassifizierung von Reviews in folgende Typen:
\begin{center}
\begin{tabular}{c|c|c}
Texttyp & introvertiert & extrovertiert \\
\hline 
emotional & stetig & initiativ\\
rational & gewissenhaft & dominant
\end{tabular}
\end{center}
\item Gegeben: 439 bereits klassifizierte Reviews
\end{itemize}
\end{frame}


%<-------------Folie 4--------->
\section{Erstellen von Prädiktoren}
\begin{frame}
\frametitle{Erstellen von Prädiktoren}
\begin{itemize}\itemsep12pt
\item Klassifikation sollte durch verwendete Wörter geschehen
\item Zurückführung auf Grundwörter notwendig
\item Benutzung verschiedener Packages in \texttt{R} bzw. \texttt{Python} ermöglichte verschiedene Verfahren
\end{itemize}
\end{frame}

\section{Lemmatisierung und Stemming}
\begin{frame}
\frametitle{Erstellen von Prädiktoren}
\framesubtitle{Was ist Stemming?}
\begin{itemize}\setlength\parskip{12pt}
	\item Verfahren, mithilfe dessen man verschiedene Varianten eines Wortes auf ihren gemeinsamen Wortstamm zurückführt
	\item Durch Abschneiden von Prä-/In- und Suffixen und Ersetzen von Umlauten, Diphtongen etc. erzeugen von Wortstämmen
	\item Beispiele: \ \ gelernt $\rightarrow$ lernen; \ \ \ Wohnungen $\rightarrow$ Wohnung
	

\end{itemize}

\end{frame}

\begin{frame}
\frametitle{Erstellen von Prädiktoren}
\framesubtitle{Stemming}
\begin{itemize}\itemsep12pt
\item Eigene Implementierung nach Vorgabe von COMPEON in \texttt{R}
\item Für Englische Sprache bereits vorgefertigte Tools z.B. 
\begin{itemize}
\item \texttt{porterstemmer} von \texttt{nltk} in Python
\item \texttt{snowballstemmer} von \texttt{nltk} in Python
\end{itemize}
\end{itemize}

\end{frame}

\begin{frame}
\frametitle{Erstellen von Prädiktoren}
\framesubtitle{Was ist Lemmatisierung?}
\begin{itemize}
\item Das Lemma ist im Bereich der Linguistik die Grundform eines Wortes $\rightarrow$ Wortform z.B. in einem Nachschlagewerk
\item Zurückführung auf grammatikalische Grundformen
\item Lemmatisierung als ein lexikonbasiertes Stemmingverfahren
\item auftretende Probleme des Vorgangs:
\begin{itemize}
\item Ambiguitäten
\item Wahl des Lemma eines Wortes (Verbinfinitiv vs. Nomen)
\item Kompositazerlegung nicht eindeutig (Beispiel: Wachstube)
\item Simplizia (Beispiel: Kreuzer, Tangente)
\item Unregelmäßigkeit von Verben im Deutschen
\end{itemize}

\end{itemize}

\end{frame}

\begin{frame}
\frametitle{Erstellen von Prädiktoren}
\framesubtitle{Lemmatisierung}
\begin{itemize}\itemsep12pt
\item Erfordert vorgefertigte Packages z.B.
\begin{itemize}
\item \texttt{SpaCy} in Python
\item \texttt{nltk} in Python
\end{itemize}
\item Diese lieferten zusätzlich Informationen über die Wortart
\item Auch hier für Englische Sprache ausgereifter als die deutsche Alternative
\end{itemize}
\end{frame}

\begin{frame}
\frametitle{Erstellen von Prädiktoren}
\framesubtitle{Wortliste}
\begin{itemize}
\item Ausgangssituation: 439 Reviews über die Firma COMPEON 
\item 8792 Wörter in reviews\_preprocessed (mitunter mehrfach)
\end{itemize}
\begin{table}
	\centering
		\begin{tabular}[h]{l|c|r}
			cleaned\_text & preprocessed\_text \\
			\hline
			richtigen & richtig \\
			\hline
			darlehen & darleh \\
			\hline
			gewünschte & wunsch \\
			\hline
			taggenuae & taggenua \\
			\hline
			gegenüber & genub
			
			
		\end{tabular}
		\item
	\caption{Wortliste Beispiele}
	\label{tab:WortlisteBeispiele}
\end{table}
\end{frame}