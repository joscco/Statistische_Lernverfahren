\subsection{Naive Bayes}
%<-------------Folie 8--------->

% Jonathans Text
\begin{frame}
\frametitle{Analysemethoden}
\framesubtitle{Naive Bayes}
\begin{itemize}\itemsep12pt
\item Das Naive Bayes Verfahren fußt auf dem Bayes Theorem
\[p(y|x) = \frac{p(x|y)p(y)}{p(x)}\]
bzw. für unabhängige Prädiktoren $x_1,...,x_n$ als
\[p(y|x_1,...,x_n)=\frac{p(x_1|y)\cdots p(x_n|y)p(y)}{p(x_1,...,x_n)}\propto p(x_1|y)\cdots p(x_n|y)p(y).\]
\item Durch Schätzen von $p(y)$ und $p(x_i|y)$ (für die Reviewtypen $y$) durch die relativen Häufigkeiten, können wir dann Klassifikationen durchführen als
\[\hat{y}=\text{argmax}_y\,p(y)\prod_{i=1}^np(x_i|y).\]
\end{itemize}
\end{frame}

\begin{frame}
\frametitle{Analysemethoden}
\framesubtitle{Naive Bayes}
\begin{itemize}\itemsep12pt
\item Durchführung war in \texttt{R} mit dem Package \texttt{caret}, über \texttt{Python} mit \texttt{sklearn} möglich. Mit letzterem haben wir jeweils die deutschen und englischen Reviews klassifiziert.
\item Dieses Vorgehen zeigte nur wenig bessere Ergebnisse als eine einheitliche Zuweisung.
\end{itemize}
\begin{center}
(Bester) Naive Bayes, \texttt{R}, Wortaufkommen $> 20$\
\begin{tabular}{|c|c|c|c|c|c|c|c|c|}
\hline
				& D 	& G	& I & S	& Acc.	& Prec. & Recall	& F1\\
\hline
Dominant 		& 14		& 2 			& 8 		& 1 		&       	& 0,412 	& 0,778 	& 0,538\\
Gewissenhaft 	& 0 		& 0 			& 0 		& 0 		& 			& n.d. 		& 0 	& n.d.\\
Initiativ 		& 4 		& 11			& 28		& 17		& 			& 0,467	& 0,778 	& 0,583\\
Stetig 			& 0 		& 1 			& 0 		& 0 		& 			& n.d.	   		& 0 	& n.d\\
\hline
Total 			& 			& 				& 			& 			& 0,494		& n.d. 		& 0,389  	& n.d.\\
\hline
\end{tabular}
\end{center}
\end{frame}

\begin{frame}
\begin{center}
Naive Bayes, \texttt{Python}, Wortvorkommen in mind. 1\% der Texte,\\
Lemmatisierung mit spacy, Englisch\
\begin{tabular}{|c|c|c|c|c|c|c|c|c|}
\hline
 & D 	& G	& I & S	& Acc.	& Prec. & Recall	& F1\\
\hline
Dominant & 13 & 4 & 12 & 4 & & 0,433 & 0,722 & 0,542\\
Gewissenhaft & 0 & 3 & 3 & 0 & & 0,5 & 0,214 & 0,3\\
Initiativ & 4 & 5 & 16 & 11 & & 0,444 & 0,444 & 0,444\\
Stetig & 1 & 2 & 5 & 3 & & 0,273 & 0,167 & 0,207\\
\hline
Total  &   &   &   &   & 0,407 & 0,413  & 0,387   & 0,373\\
\hline
\end{tabular}
\end{center}
\end{frame}

\begin{frame}
\begin{center}
Naive Bayes, \texttt{Python}, Wortvorkommen in mind. 1\% der Texte,\\
Lemmatisierung mit spacy, Deutsch\
\begin{tabular}{|c|c|c|c|c|c|c|c|c|}
\hline
 & D 	& G	& I & S	& Acc.	& Prec. & Recall	& F1\\
\hline
Dominant & 16 & 2 & 13 & 2 & & 0,444 & 0,889 & 0,593\\
Gewissenhaft & 0 & 5 & 4 & 1 & & 0,5 & 0,357 & 0,417\\
Initiativ & 2 & 5 & 16 & 13& & 0,444 & 0,444 & 0,444\\
Stetig & 0 & 2 & 3 & 2& & 0,286 & 0,111 & 0,16 \\
\hline
Total  &   &   &   &   & 0,453 & 0,419  & 0,45   & 0,403\\
\hline
\end{tabular}
\end{center}
\end{frame}